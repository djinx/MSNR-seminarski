% !TEX encoding = UTF-8 Unicode

\documentclass[a4paper]{article}

\usepackage{color}
\usepackage{url}
\usepackage[T2A]{fontenc} % enable Cyrillic fonts
\usepackage[utf8]{inputenc} % make weird characters work
\usepackage{graphicx}

\usepackage[english,serbian]{babel}
%\usepackage[english,serbianc]{babel} %ukljuciti babel sa ovim opcijama, umesto gornjim, ukoliko se koristi cirilica

\usepackage[unicode]{hyperref}
\hypersetup{colorlinks,citecolor=green,filecolor=green,linkcolor=blue,urlcolor=blue}

%\newtheorem{primer}{Пример}[section] %ćirilični primer
\newtheorem{primer}{Primer}[section]

\begin{document}

\title{Naslov seminarskog rada\\ \small{Seminarski rad u okviru kursa\\Metodologija stručnog i naučnog rada\\ Matematički fakultet}}

\author{Prvi autor, drugi autor (treći autor)\\ kontakt email prvog, drugog (trećeg) autora}
\date{9.~april 2015.}
\maketitle

\abstract{
U ovom tekstu je ukratko prikazana osnovna forma seminarskog rada. Obratite pažnju da je pored ove .pdf datoteke, u prilogu i odgovarajuća .tex datoteka, kao i .bib datoteka korišćena za generisanje literature. Na prvoj strani seminarskog rada su naslov, apstrakt i sadržaj, i to sve mora da stane na prvu stranu! Kako bi Vaš seminarski zadovoljio standarde i očekivanja, koristite uputstva i materijale sa predavanja na temu pisanja seminarskih radova. Ovo je samo šablon koji se odnosi na fizički izgled seminarskog rada (šablon koji \emph{morate} da ispoštujete!) kao i par tehničkih pomoćnih uputstava. Molim Vas da kada budete predavali seminarski rad, imenujete datoteke tako da sadrže temu seminarskog rada, kao i imena i prezimena članova grupe (ili samo temu i prezimena, ukoliko je sa imenima predugačko). Predaja seminarskih radova biće isključivo preko web forme, a NE slanjem mejla.

\tableofcontents

\newpage

\section{Uvod}
\label{sec:uvod}

Ко жели, може да пише рад ћирилицом. У том случају, неопходно је да су инсталирани одговарајући пакети: texlive-fonts-extra, texlive-latex-extra, texlive-lang-cyrillic, texlive-lang-other. \\

Uz sve novouvedene termine u zagradi naglasiti od koje engleske reči termin potiče. Naredni primeri ilustruju način uvođenja enlegskih termina kao i citiranje.

\begin{primer}
Problem zaustavljanja (eng.~{\em halting problem}) je neodlučiv \cite{haltingproblem}.
\end{primer}

\begin{primer}
Za prevođenje programa napisanih u programskom jeziku C može se koristiti GCC kompajler \cite{gcc}.
\end{primer}

\begin{primer}
 Da bi se ispitivala ispravost softvera, najpre je potrebno precizno definisati njegovo ponašanje \cite{laski2009software}. 
\end{primer}

Reference koje se koriste u ovom tekstu zadate su u datoteci {\em seminarski.bib}. Prevođenje u pdf format u Linux okruženju može se uraditi na sledeći način:
\begin{verbatim}
pdflatex TemaImePrezime.tex 
bibtex TemaImePrezime.aux 
pdflatex TemaImePrezime.tex 
pdflatex TemaImePrezime.tex 
\end{verbatim}
Prvo latexovanje je neophodno da bi se generisao {\em .aux} fajl. {\em bibtex} proizvodi odgovarajući {\em .bbl} fajl koji se koristi za generisanje literature. 
Potrebna su dva prolaza (dva puta pdflatex) da bi se reference ubacile u tekst (tj da ne bi ostali znakovi pitanja umesto referenci). Dodavanjem novih referenci potrebno je ponoviti ceo postupak.  


Broj naslova i podnaslova je proizvoljan. Neophodni su samo Uvod i Zaključak. Na poglavlja unutar teksta referisati se po potrebi. 
\begin{primer}
U odeljku \ref{sec:naslov1} precizirani su osnovni pojmovi, dok su zaključci dati u odeljku \ref{sec:zakljucak}.
\end{primer}

Još jednom da napomenem da nema razloga da pišete:
\begin{verbatim}
\v{s} i \v{c} i \'c ...
\end{verbatim}
Možete koristiti srpska slova
\begin{verbatim}
š i č i ć ... 
\end{verbatim}


Ovde pišem uvodni tekst.
Ovde pišem uvodni tekst. 
Ovde pišem uvodni tekst. 
Ovde pišem uvodni tekst. 


\section{Slike i tabele}
\label{slike_i_tabele}

Slike i tabele treba da budu u svom okruženju, sa odgovarajućim naslovima, obeležene labelom da koje omogućava referenciranje. 

\begin{primer} Ovako se ubacuje slika. Obratiti pažnju da je dodato i 
\begin{verbatim}
\usepackage{graphicx}
\end{verbatim}

\begin{figure}[h!]
\begin{center}
%\includegraphics[scale=0.75]{panda.jpg}
\end{center}
\caption{Pande}
\label{fig:pande}
\end{figure}

Na svaku sliku neophodno je referisati se negde u tekstu. Na primer, na slici \ref{fig:pande} prikazane su pande. 
\end{primer}

\begin{primer} I tabele treba da budu u svom okruženju, i na njih je neophodno referisati se u tekstu. Na primer, u tabeli \ref{tab:tabela1} su prikazana različita poravnanja u tabelama.

\begin{table}[h!]
\begin{center}
\caption{Razlčita poravnanja u okviru iste tabele ne treba koristiti jer su nepregledna.}
\begin{tabular}{|c|l|r|} \hline
centralno poravnanje& levo poravnanje& desno poravnanje\\ \hline
a &b&c\\ \hline
d &e&f\\ \hline
\end{tabular}
\label{tab:tabela1}
\end{center}
\end{table}

\end{primer}





\section{Prvi naslov}
\label{sec:naslov1}


Ovde pišem tekst. 
Ovde pišem tekst. 
Ovde pišem tekst. 
Ovde pišem tekst. 
Ovde pišem tekst. 
Ovde pišem tekst. 
Ovde pišem tekst. 
Ovde pišem tekst. 


\subsection{Prvi podnaslov}
\label{subsec:podnaslov1}

Ovde pišem tekst. 
Ovde pišem tekst. 
Ovde pišem tekst. 
Ovde pišem tekst. 
Ovde pišem tekst. 
Ovde pišem tekst. 
Ovde pišem tekst. 

\subsection{Drugi podnaslov}
\label{subsec:podnaslov2}

Ovde pišem tekst. 
Ovde pišem tekst. 
Ovde pišem tekst. 
Ovde pišem tekst. 
Ovde pišem tekst. 
Ovde pišem tekst. 

\section{Drugi naslov}
\label{sec:naslov2}

Ovde pišem tekst. 
Ovde pišem tekst. 
Ovde pišem tekst. 
Ovde pišem tekst. 

\subsection{... podnaslov}
\label{subsec:podnaslovN}

Ovde pišem tekst. 
Ovde pišem tekst. 
Ovde pišem tekst. 
Ovde pišem tekst. 
Ovde pišem tekst. 
Ovde pišem tekst. 

\section{n-ti naslov}
\label{sec:naslovN}

Ovde pišem tekst. 
Ovde pišem tekst. 
Ovde pišem tekst. 
Ovde pišem tekst. 
Ovde pišem tekst. 

\subsection{... podnaslov}
\label{subsec:podnaslovK}

Ovde pišem tekst. 
Ovde pišem tekst. 
Ovde pišem tekst. 
Ovde pišem tekst. 
Ovde pišem tekst. 

\subsection{... podnaslov}
\label{subsec:podnaslovM}

Ovde pišem tekst. 
Ovde pišem tekst. 
Ovde pišem tekst. 
Ovde pišem tekst. 
Ovde pišem tekst. 

\section{Poslednji naslov}
\label{sec:naslovM}

Ovde pišem tekst. 
Ovde pišem tekst. 
Ovde pišem tekst. 
Ovde pišem tekst. 
Ovde pišem tekst. 
Ovde pišem tekst. 
Ovde pišem tekst. 
Ovde pišem tekst. 
Ovde pišem tekst. 

% TODO: pročitajte i dajte svoj sud
% ovo poglavlje je nezvanicno završeno
% potencijalno treba da se dopuni ili lepše zapiše da bi sve delovalo kao jedna celina

\section{Polimorfna provera tipova}
\label{sec:provera tipova}


%The Implementation of Functional Programming Languages strana 139
Neki moderni jezici, kao što je Miranda, imaju svojstvo koje omogućava programeru da ne navodi tipove objekata koje definiše u programu. Kompilator može da odredi tipove ako je to moguće. Deo kompilatora koji se bavi ovim poslom naziva se \textit{zaključivač tipova} \cite{the-implementation-of-functional-programming-languages}. Proveravač tipova je od velike koristi programeru jer mu ukazuje na greške, od trivijalnih propusta u kucanju do velikih logičkih grešaka. Pomaže u pisanju robusnih programa kao i u izgradnji bržih implementacija programskih jezika. Ako zaključivač tipova obradi program, pri izvršavanju se neće javiti greške poput upotrebe promenljive tipa bool kao da je tipa int.
\\
\\ %BasicTypechecking.pdf str. 6
% potrebna referenca za unifikaciju

Izrazi koji sadrže nekoliko pojavljivanja istog tipa, kao $\alpha \longrightarrow \alpha$, izražavaju kontekstnu zavisnost, u ovom slučaju to je zavisnost domena i kodomena tipa funkcije. Proces zaključivanja tipova sastoji se od uparivanja tipova operatora i instanciranja tipova promenljivih. Kad god se tip promenljive instancira, sve ostale pojave iste promenljive moraju biti instancirane sa istom vrednošću: ispravna instanciranja izraza $\alpha \longrightarrow \alpha$ su $int \longrightarrow int$,  $bool \longrightarrow bool$, itd. Proces kontekstnog instanciranja izvodi se pomoću \textit{unifikacije} i ona je osnova polimorfne provere tipova. Unifikacija ne uspeva kada pokušava da upari dva operatora različitih tipova (npr. int i bool) ili kada pokušava da instancira promenljivu izrazom koji sadrži tu promenljivu (npr. $a$ i $a\longrightarrow b$, gde će se napraviti rekurzija bez izlaza) \cite{basic-typechecking}. %The latter situation arises in typechecking self-application (e.g. fun(x) x(x)), which is therefore considered illegal.
U opštem slučaju, tip izraza određuje se pomoću skupa pravila kombinovanja tipova za jezičke konstrukcije i tipova primitivnih operatora. 

\iffalse
%The Implementation of Functional Programming Languages sekcija 8.1
\subsection{Ukratko o notaciji}
%ovo je sve nebitno zapravo, ali neka stoji za sad :D
Tipovi koji su zanimljivi kada je funkcionalno programiranju u pitanju su karakteri, broj, istinitosna vrednost, kao i tipovi torki, lista i funkcija. Kada govorimo o ovim tipovima, koristićemo sledeću notaciju:
$$a::A$$

\noindent što predstavlja promenljivu $a$ koja je tipa $A$. 	
\\
\\ Ako su dati tipovi $A_1, \ldots, A_n$ onda $(A_1, \ldots, A_n)$ predstavlja tip tokre $(a_1, \ldots, a_n)$ za koji važi $a_1::A_1$ i $a_n::A_n$. Bitno je naglasiti da $A_1, \ldots, A_n$ ne predstavljaju iste tipove, odnosno da koordinate jedne torke ne moraju biti sve istog tipa. Takođe, tip torke određuje broj koordinata (odnosno dimenziju torke) i njihove tipove. 
\\
\\ Ako je dat tip $A$ onda je $[B]$ tip liste čiji su elementi tipa $B$. U slučaju da su elementi liste torke, sve torke moraju biti istog tipa. Za razliku od tipa torke, tip liste ne određuje njenu dužinu. 
\\
\\ Ako su dati tipovi $A$ i $B$ koristimo $A \longrightarrow B$ za zapis tipa funkcije $f$ koja se primenjuje na promenljivu $a::A$, a čije vrednosti $(f \quad a)$ su tipa $B$.


\fi 


%BasicTypechecking.pdf str. 9
\subsection{Zaključivanje tipova}
\label{subsec: zakljucivanje tipova}

Osnovni algoritam za zaključivanje tipova opisan je u nastavku \cite{basic-typechecking}.

\begin{enumerate}
	\item Kada se pojavi nova promenljiva $x$, njoj se dodeljuje novi tip promenljive što znači da joj tip mora biti određen u daljem kontekstu u kom se pojavljuje. Par $<x, a>$ se čuva u okruženju koje se pretražuje svaki put kad se pojavi $x$, u kom je $x$ tipa $a$.
	
	\item Kad imamo uslovno grananje, izraz u \textit{if} se uparuje sa bool, dok se \textit{then} i \textit{else} grane ostavljaju nedefinisane kako bi se odredio jedinstven tip za ceo izraz.
	
	\item U apstrakciji $\lambda x.e$, tip za $e$ se zaključuje u kontekstu gde je $x$ povezan sa novim tipom promenljive.
	
	\item U aplikaciji $f(a)$, tip od f se unifikuje sa tipom $A \longrightarrow b$, gde je $A$ tip parametra $a$, dok je $b$ nova tipska promenljiva. Ovo ukazuje na to da $f$ mora biti tipa funkcije čiji domen se unifikuje sa $A$, a $b$ je tip povratne vrednosti.
\end{enumerate}





\section{Zaključak}
\label{sec:zakljucak}

Ovde pišem zaključak. 
Ovde pišem zaključak. 
Ovde pišem zaključak. 
Ovde pišem zaključak. 
Ovde pišem zaključak. 
Ovde pišem zaključak. 
Ovde pišem zaključak. 
Ovde pišem zaključak. 
Ovde pišem zaključak. 
Ovde pišem zaključak. 
Ovde pišem zaključak. 
Ovde pišem zaključak. 



\addcontentsline{toc}{section}{Literatura}
\appendix
\bibliography{seminarski} 
\bibliographystyle{plain}

\appendix
\section{Dodatak}
Ovde pišem dodatne stvari, ukoliko za time ima potrebe.
Ovde pišem dodatne stvari, ukoliko za time ima potrebe.
Ovde pišem dodatne stvari, ukoliko za time ima potrebe.
Ovde pišem dodatne stvari, ukoliko za time ima potrebe.
Ovde pišem dodatne stvari, ukoliko za time ima potrebe.



\end{document}
