\section{G-mašina}
\label{sec:Gmasine}

U ovom delu predstavićemo {\em G mašinu} (engl. \textit{G-machine}), koju su Avgustson (engl. Lennart Augustsson) i Džonson (engl.Thomas Johnsson) predstavili u svojim naučnim radovima, na Tehnološkom institutu Čalmers (engl. Chalmers Institute of Technology (\url{http://www.chalmers.se/en})) u Geteburgu, između 1984. i 1987. godine. 

\subsection{Motivacija}

Najčešća razlika između kompilatora za imperativne i funkcionalne programske jezike je u tome što se imperativni programski jezici direktno prevode na mašinski jezik, dok se funkcionalni programski jezici ili interpretiraju, ili se prevode na neki međujezik koji se izvršava na virtuelnoj mašini. Razlog prevođenja na međukod leži u tome što mehanizmi funkcionalnih programskih jezika, kao što su: rekurzija, zatvorenja i ostali, ne mogu da se implementiraju neposredno na mašinskom jeziku.

Kod lenjih funkcionalnih jezika, funkcije i argumenti konstruktora se evaluiraju samo po potrebi, pa čak i onda najviše samo jednom. Bez obzira na to što se ovako nešto može implementirati predstavljanjem neevaluiranih argumenata \textit{tankovima} (engl. \textit{thunk}), tj. funkcijom koja evaluira argument i zamenjuje ga rezultatom, zbog efikasnosti biramo drugačije pristupe.

Osnovna ideja je da predstavimo program grafom koji bismo kasnije ‚‚prepisali''  evaluacijom. Evaluaciju vršimo za neki podizraz, a potom u grafu zamenimo sva pojavljivanja tog podizraza odgovarajućom evaluacijom i nastavljamo dalje evaluiranje. Velika prednost je u tome što evaluacija deljenog podgrafa automatski razrešava sve izraze koji pokazuju na njega. Međutim, treba biti pažljiv jer je kontinuirana pretraga grafa za podizrazima spora.

Rane implementacije su prevodile program u fiksirani skup \textit{kombinatora} (engl. \textit{combinator}), odnosno, zatvorene lambda izraze koje možemo posmatrati kao pravila za ‚‚prepisivanje grafa''. Kasnije se pokazalo da je korisno prepustiti programu da razmatra izbor kombinatora, tzv. {\em superkombinatore} (engl. \textit{supercombinators}) \cite{super-combinators}. To znači da su superkombinatori ‚‚pametniji'' izbor kombinatora koje ćemo vršiti prepisivanje grafa. 

\subsection{Osnovna ideja} 

Umesto da interpretiraju superkombinatore kao pravila za prepisivanje grafa, oni se kompiliraju u k\^ od sa specijalnim instrukcijama za manipulaciju grafom. G-mašine su osnova implementacije lenjog ML (engl. \textit{Lazy ML}) \cite{lazy-ML} i Haskela \cite{hbc}. 

Osnovna ideja koja stoji iza G-mašina je da se funkcionalni kod prevodi u superkombinatore, pa se oni prevode preko međukoda u mašinski k\^ od. Pre pokretanja programa svako telo superkombinatora prevede u niz instrukcija koje, nakon što se izvrše, kreiraju instancu tela superkombinatora. Svaki put samo instanciramo superkombinator, da bismo mogli da odbacimo originalne superkombinatore nakon što je prevođenje završeno, i sačuvamo samo preveden k\^ od. Dakle, koristimo kompilator G-mašine da izvorni k\^ od prevedemo u niz instrukcija na mašinskom jeziku.

Nakon što smo preveli telo superkombinatora u niz instrukcija, moramo da odlučimo u koji jezik ćemo ga prevesti. Najbolje rešenje, ujedno i ono koje se najčešće koristi, jeste da prevođenje izvršimo u \textit{međuk\^ od} (engl. \textit{intermediate code}) za G-mašinu, takozvani {\em G-k\^ od} (engl. \textit{G-code}).

\subsection{G-k\^ od}

G-k\^ od je k\^ od u koji kompiliramo tela superkombinatora. Kompilator za G-mašinu prati sledeći niz postupaka \cite{the-implementation-of-functional-programming-languages, abstract-machines}:

\begin{enumerate}
	\item Izvorni jezik je varijanta ML-a, sa semantikom lenje evaluacije, tzv. {\em lenji ML}.
	\item Rane faze kompilacije vrše proveru tipova, nalaženje uzoraka i analizu zavisnosti. U ovoj fazi program je preveden na lambda račun.
	\item \textit{Lambda-podizač} (engl. \textit{lambda-lifter}) transformiše program u oblik superkombinatora.   
	\item Superkombinatori se prevode u G-k\^ od.
	\item Konačno, generiše se mašinski k\^ od iz G-k\^ oda za ciljnu mašinu.
\end{enumerate}


\begin{primer}
	Predstavimo sad način na koji funkcioniše kompilator za G mašinu razmatrajući sledeću funkciju:
	\begin{center}
		\verb|f g x = K (g x)|.
	\end{center}
	Ova funkcija bi se prevela u sledeći niz instrukcija G-k\^ oda (svaka instrukcija je odvojena simbolom tačka-zapeta):
	\begin{center}
		\verb|Push 1|; \verb|Push 1|; \verb|Mkap|; \verb|Pushglobal K|; \verb|Mkap|; \verb|Slide 3|; \verb|Unwind|.
	\end{center}
\end{primer}


\begin{figure}[h!]
	\centering
	\includegraphics[width=0.8\textwidth]{primerGmasine.png}
	
	\caption{Vizuelni prikaz izvršavanja k\^ oda.}
	\label{fig:primerGmasine}
\end{figure}

Na slici \ref{fig:primerGmasine} prikazujemo kako se k\^ od izvršava. Sa leve strane svakog dijagrama prikazan je stek, koji raste nadole. Ostatak svakog dijagrama je hip. Aplikativni čvorovi predstavljeni su karakterom \verb|@|, izrazi malim slovima, a superkombinatori velikim slovima latinice. Takođe, prikazano je stanje mašine pre izvođenja instrukcija za \verb|f|. Dva elementa na vrhu steka su pokazivači na aplikativne čvorove, čije su desne strane izrazi koji će biti vezani za \verb|g| i \verb|x|. Instrukcija \verb|Push| koristi relativno adresiranje u odnosu na vrh steka. Ignorišući pokazivač na supekombinatorski čvor \verb|f|, prvi element na steku dobija broj \verb|0|, naredni \verb|1|, itd.

Naredni dijagram (b) pokazuje kako se stek izmenio nakon primene \verb|Push 1| instrukcije. Ona stavlja pokazivač na izraz \verb|x| na stek. Nakon jos jednog izvršavanja \verb|Push 1| instrukcije, imamo pokazivač na \verb|g| na vrhu steka i onda imamo ono što je prikazano na dijagramu (c). Dijagram (d) pokazuje šta se dešava kada se izvrši \verb|Mkap| instrukcija. Ona uzima dva pokazivača sa steka i pravi aplikativni čvor, ostavljajući pokazivač na rezultat na steku. Na dijagramu (e) izvršavamo \verb|Pushglobal K| instukciju koja stavlja pokazivač na \verb|K| superkombinator. Na dijagramu (f) vidimo da još jedna \verb|Mkap| instrukcija završava instanciranje tela \verb|f|. 

Sada možemo zameniti originalni izraz \verb|f g x| novoinstanciranim telom \verb|K(g x)|. U prvoj (ne lenjoj) verziji G-mašine, jednostavno pomerimo telo nadole za 3 mesta na steku pomoću instrukcije \verb|Slide 3| (dijagram (g)). Konačno, \verb|Unwind| instrukcija uzrokuje dalju evaluaciju.


