\section{Efikasan k\^od}
\label{sec:efikasan kod}

% Compilation_of_Functional_Languages_by_Program_Tra.pdf

Jedan od najvažnijih problema vezanih za funkcionalne jezike jeste efikasnost izvršavanja. 
U nastavku opisujemo tehnike koje se koriste za generisanje efikasnog k\^oda.	


\subsection{Transformacije lambda računa}

Već smo napomenuli da se lambda račun koristi zbog svoje jednostavnosti, kao i da je dovoljno izražajan da se bilo koji funkcionalan jezik može transformisati u njega. To znači da ako imamo implementaciju lambda računa, možemo da implementiramo bilo koji funkcionalan jezik tako što ćemo ga prevesti u lambda račun. Transformacije se često grupišu u dva skupa \cite{compilation-by-program-transformation}:

\begin{itemize}
	\item Veliki skup jednostavnih, lokalnih transformacija koje su implementirane u okviru dela kompilatora koji nazivamo \textit{pojednostavljivač} (engl. \textit{simplifier}). Složenost programa raste jer pojednostavljivač pokušava da izvrši što je moguće više transformacija u jednom prolasku kroz program. Uprkos tome, rezultat nakon tih transformacija često i dalje sadrži delove koji mogu biti dalje transformisani, tako da se ova operacija ponavlja više puta, dokle god ima nešto da se pojednostavi. Transformacije koje spadaju u ovaj skup su simplifikacija (engl. \textit{simplification}) \cite{compilation-by-program-transformation}, $\beta$-odsecanje, umetanje (engl. \textit{inlining}) i \textit{sklapanje konstanti} (engl. \textit{constant folding}).
	
	\item Manji skup složenijih, globalnih transformacija, od kojih se svaka implementira odvojeno. Mnoge se sastoje od faze analize praćene transformacijama koje koriste rezultate analize za otkrivanje pogodnih delova za transformisanje. Mnogi se oslanjaju na to da će pojednostavljivač ‚‚počistiti'' k\^ od koji proizvedu, tako da se izbegava ponavljanje transformacija koje su već sadržane u telu pojednostavljivača. Primer transformacije za ovaj skup je  \textit{analiza strogosti} (engl. \textit{strictness analasys}) \cite{haskell-by-program-transformation}. 
\end{itemize}

\subsection{Umetanje}
% inline.pdf str. 2
Umetanje funkcijskih poziva je dobar metod za unapređivanje performansi programa koji se sastoji iz malih funkcija, kao što je čest slučaj sa funkcionalnim jezicima. Osnovni princip umetanja je sledeći: poziv funkcije zamenjuje se njenim telom. Umetanje uklanja neke pozive funkcija, ali jednako je važno da spaja k\^ od koji je prethodno razdvojen i zbog toga se implementira u okviru pojednostavljivača \cite{compilation-by-program-transformation}.

Korisno je definisati tri različite transformacije koje zajedno implementiraju ono što neformalno opisujemo kao ‚‚umetanje'' \cite{secrets-haskell-compiler-inliner, compilation-by-program-transformation}:
\begin{itemize}
	\item \textit{Samo umetanje} (engl. \textit{inlining itself}) zamenjuje sva pojavljivanja vezane promenljive kopijom desne strane njene definicije. Pogledajmo to na primeru.
	\begin{primer} ~
		\begin{center}
			\verb|let {f = |$\lambda$\verb|x.x*4} in (f (a*b - c)) + a*d| \\
			$\stackrel{\text{inline f}}{\longrightarrow}$ \verb|let {f = |$\lambda$\verb|x.x*4} in ((|$\lambda$\verb|x.x*4) (a*b - c)) + a*d|
		\end{center}
	\end{primer}
	
	\item \textit{Eliminacija mrtvog koda} (engl. \textit{dead code elimination}) odbacuje veze koje se više ne koriste. Ovo se najčešće dešava kada je svaka pojava promenljive umetnuta. Nadovežimo se na prethodni primer.
	\begin{primer} ~
		\begin{center}
			\verb|let {f = |$\lambda$\verb|x.x*4} in ((|$\lambda$\verb|x.x*4) (a*b - c)) + a*d| \\
			$\stackrel{\text{dead f}}{\longrightarrow}$ \verb|((|$\lambda$\verb|x.x*4) (a*b - c)) + a*d| 
		\end{center}
	\end{primer}
	
	\item \textit{$\beta$-odsecanje} (engl. \textit{$\beta$-reduction}) jednostavno prezapiše lambda izraz \verb|((|$\lambda$\verb|x.E) A)| u \verb|(let {x = A} in E)|. Nadovežimo se na prethodni primer.
	\begin{primer} ~
		\begin{center}
			\verb|((|$\lambda$\verb|x.x*4) (a*b - c)) + a*d| \\
			$\stackrel{\beta}{\longrightarrow}$ \verb|(let {x = a*b - c} in x*4) + a*d| 
		\end{center}
	\end{primer}
\end{itemize}

\subsection{Uparivanje šablona}

Funkcije nad običnim tipovima, kao i nad strukturiranim algebarskim tipovima \cite{algebraic-types}, mogu se definisati preko različitih \textit{slučajeva} (engl. \textit{case}). Različiti slučajevi dati su \textit{šablonima} (engl. \textit{pattern}), koji se uparuju prema vrednostima tog tipa. Ako šablon odgovara vrednosti, promenljive koje se mogu pojaviti u šablonu su vezane za odgovarajuće promenljive komponente te vrednosti. Pogledajmo kako to radi na jednostavnom primeru. 


\begin{primer}
	Neka imamo šablone \verb|[1; x; y]| $\longrightarrow$ \verb|x*y| i \verb|[z]| $\longrightarrow$ \verb|z|, kao i izraz \verb|[1; 2; 3]|. Vidimo da se dati izraz poklapa sa prvim šablonom. Promenljive \verb|x| i \verb|y| vezuju se za komponente 2 i 3 i vraća se njihov proizvod.
\end{primer}


\begin{primer}
	Računanje $n$-tog fibonačijevog broja u programskom jeziku Haskell dato je k\^odom:
	\begin{verbatim}
	fibonaci n
	|n==0 = 1
	|n==1 = 1
	|otherwise = (fibonaci (n-1))+(fibonaci (n-2))
	\end{verbatim}
	Prilikom evaluacije \verb|(fibonaci x)|, imamo tri slučaja od kojih će jedan biti odgovarajući: prvi je kada je vrednost parametra \verb|x| jednaka \verb|0|, drugi je kada je vrednost parametra \verb|x| jednaka \verb|1|, a treći kada nisu ispunjena prva dva uslova. Uslovi se isprobavaju jedan po jedan, od vrha, dok se ne pronađe odgovarajući šablon \cite{the-implementation-of-functional-programming-languages}.
\end{primer}

Ovakvo poređenje na osnovu šablona naziva se \textit{uparivanje šablona} (engl. \textit{pattern matching}) \cite{compiler-design}.

Uparivanje šablona je zapravo skraćenica za ugnežđene \verb|if-then-else| i \verb|case| naredbe (engl. \textit{statements}): pokušaj da upariš ulaz sa prvim šablonom i ako se poklapa izvrši odgovarajuću akciju, u suprotnom se vrati korak nazad i pokušaj sa sledećim šablonom. Ova strategija proizvodi nepotreban rad. Kompilatori to mnogo bolje rade tako što šablone kompiliraju u \textit{diskriminišuće stablo} (engl. \textit{discrimination tree}) koje to radi u skoro optimalnom broju testova nad podacima kako bi odabralo odgovarajuću akciju \cite{compiling-fl}.

