\section{Polimorfna provera tipova}
\label{sec:provera tipova}

Neki moderni jezici, kao što je Miranda, imaju svojstvo koje omogućava programeru da ne navodi tipove objekata koje definiše u programu. Kompilator može da odredi tipove ako je to moguće. Deo kompilatora koji se bavi ovim poslom naziva se \textit{zaključivač tipova} (engl. \textit{type-checker}) \cite{the-implementation-of-functional-programming-languages}. Zaključivač tipova je od velike koristi programeru jer mu ukazuje na greške, od trivijalnih propusta u kucanju do velikih logičkih grešaka. Pomaže u pisanju robusnih programa kao i u izgradnji bržih implementacija programskih jezika. Ako zaključivač tipova obradi program, pri izvršavanju se neće javiti greške poput upotrebe promenljive tipa \verb|bool| kao da je tipa \verb|int|.

%NJEN KOMENTAR: type checker je proveravac a ne zakljucivac tipova, mada vi u tom delu 
%pricate o zakljucivanju a ne o proveri tipova. Treba vezu izmedju 
%proveravanja tipova i zakljucivanja tipova bolje objasniti.
%MOJ DODATAK: I dalje nisam izmenila naziv jer si i ti rekao da je bolje da bude zaključival a ne proveravač. A i po tome što je napisala rekla bih da ne mora da se menja. Za vezu sam dodala ovaj pasus.


Provera tipova je provera da li izraz ima određen tip. Ako je dat izraz \verb|e| i tip \verb|t| jednostavno se može odrediti da li je izraz \verb|e| tipa \verb|t| primenom nekih sintaksnih pravila. Zaključivanje tipova određuje tip promenljive za koju nije naveden tip na osnovu svih konteksta u kojima se promenljiva pojavljuje. Ono što imamo jeste sistem u kom nije potrebno eksplicitno navoditi tipove izraza i promenljivih, ali pogrešna upotreba tipa će i dalje biti pronađena. Dakle, zaključivač tipova nam omogućava proveru tipova. 


Izrazi koji sadrže nekoliko pojavljivanja istog tipa, kao $\alpha \longrightarrow \alpha$, izra\-ža\-vaju kontekstnu zavisnost; u ovom slučaju to je zavisnost domena i kodomena tipa funkcije. Proces zaključivanja tipova sastoji se od uparivanja tipova operatora i instanciranja tipova promenljivih. Kad god se tip promenljive instancira, sve ostale pojave iste promenljive moraju biti instancirane istom vrednošću: ispravna instanciranja izraza $\alpha \longrightarrow \alpha$ su \verb|int| $\longrightarrow$ \verb|int|,  \verb|bool| $\longrightarrow$ \verb|bool|, itd. Proces kontekstnog instanciranja izvodi se pomoću \textit{unifikacije} (engl. \textit{unification}) i ona je osnova polimorfne provere tipova. Unifikacija ne uspeva kada pokušava da upari dva operatora različitih tipova (na primer, \verb|int| i \verb|bool|) ili kada pokušava da instancira promenljivu izrazom koji sadrži tu promenljivu (na primer, \verb|a| i \verb|a| $\longrightarrow$ \verb|b|, gde će se napraviti rekurzija bez izlaza) \cite{basic-typechecking}. U opštem slučaju, tip izraza određuje se pomoću skupa pravila kombinovanja tipova za jezičke konstrukcije i tipova primitivnih operatora. 

\subsection{Zaključivanje tipova}
\label{subsec: zakljucivanje tipova}

Osnovni algoritam za zaključivanje tipova opisan je u nastavku \cite{basic-typechecking}.

\begin{enumerate}
	\item Kada se pojavi nova promenljiva \verb|x|, njoj se dodeljuje novi tip promenljive što znači da joj tip mora biti određen u daljem kontekstu u kom se pojavljuje. Par \verb|<x, a>| se čuva u okruženju koje se pretražuje svaki put kad se pojavi \verb|x|, u kom je \verb|x| tipa \verb|a|.
	
	%NJEN KOMENTAR: objasniti odakle if u ovom kontekstu --- ne bi trebalo da je to 
	%uslovno granjanje kao u imeprativnim jezicima, vec bi trebalo da 
	%odgovara patern matching-u, to treba napomenuti
	%MOJ DODATAK: poslednje dve recenice u stavki
	\item Kad imamo uslovno grananje, izraz u \verb|if| se uparuje sa \verb|bool|, dok se \verb|then| i \verb|else| grane ostavljaju nedefinisane kako bi se odredio jedinstven tip za ceo izraz. Primetimo da \verb|if-then-else| kod funkcionalnih jezika nema isto značenje kao kod imperativnih jezika. U ovom slučaju, u pitanju je uparivanje šablona.
	
	\item U apstrakciji \verb|(|$\lambda$\verb|x.E)|, tip za \verb|E| se zaključuje u kontekstu gde je \verb|x| povezan sa novim tipom promenljive.
	
	\item U aplikaciji \verb|(f a)|, tip od \verb|f| se unifikuje sa tipom \verb|A| $\longrightarrow$ \verb|b|, gde je \verb|A| tip parametra \verb|a|, dok je \verb|b| nova tipska promenljiva. Ovo ukazuje na to da \verb|f| mora biti tipa funkcije čiji domen se unifikuje sa \verb|A|, a \verb|b| je tip povratne vrednosti.
\end{enumerate}
